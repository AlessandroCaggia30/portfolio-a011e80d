\documentclass[9pt,a4paper,twoside]{rho-class/rho}
\usepackage[english]{babel}
\let\bibhang\relax
%\usepackage{natbib}

\setbool{rho-abstract}{true} % Set false to hide the abstract
\setbool{corres-info}{true} % Set false to hide the corresponding author 
\usepackage{xcolor}
\usepackage{soul}
\definecolor{lightblue}{RGB}{135, 206, 250}
\usepackage{graphicx}  % in preamble
\usepackage{fancyhdr}
\usepackage{changepage}  % or geometry
\usepackage{graphicx}    % for resizebox
\usepackage{xcolor}


%----------------------------------------------------------
% TITLE
%----------------------------------------------------------

\title{Structural Models}

%----------------------------------------------------------
% AUTHORS AND AFFILIATIONS
%----------------------------------------------------------

\author{Alessandro Caggia}

%----------------------------------------------------------
% DATES
%----------------------------------------------------------

\dates{June 2025}

%----------------------------------------------------------
% FOOTER INFORMATION
%----------------------------------------------------------

\institution{Bocconi University}
\theday{June 04, 2026} %\today

%----------------------------------------------------------
% ABSTRACT
%----------------------------------------------------------


\begin{abstract}
\end{abstract}



\newcommand{\citeyearcomma}[1]{\citeauthor{#1}, \citeyear{#1}}

\begin{document}
	
    \maketitle
    \thispagestyle{plain}
    \linenumbers


%----------------------------------------------------------






\section{Causality in Structural Models}

\subsection{General Setup}

\begin{itemize}
  \item Say that after solving a model comprising a system of equaitons (eg supply and Demand funcitons) you end up with an equation for $Y$, say $Y = F(x_1, x_2, \dots, x_n)$
  \item Now we are interested in the \textbf{Causal effect}, the response of $Y$ to a \textit{ceteris paribus} change in one variable. If we can vary the ``cause" $x_j$ independently, then:
  \[
  \frac{\partial Y}{\partial x_j} = F_j'(x_1, x_2, \dots, x_n)
  \]
  \item This is the marginal causal effect of $x_j$ on $Y$.
  \item If we assume linear equation: $Y = \beta X + \varepsilon$, we can use OLS to recover the parameter of interest. 
  \item Under proper identification, $\beta_j = \frac{\partial \mathbb{E}[Y]}{\partial x_j}$ can be interpreted causally and $\beta_j$ is called a structural parameter as it is derived from a structural model.
\end{itemize}

\subsection{Interrelated Causes and Endogeneity}

\begin{itemize}
  \item When causes are interrelated, causal effect still exists but is harder to identify.
  \item Example: Demand and Supply
  \[
  \begin{aligned}
  Y^d &= Y^d(P^d, Z^d, U^d) \\
  Y^s &= Y^s(P^s, Z^s, U^s)
  \end{aligned}
  \]
  \item If $P^d$ and $P^s$ could be varied exogenously:
  \[
  \frac{\partial Y^d}{\partial P^d}, \quad \frac{\partial Y^s}{\partial P^s}
  \]
  would be causal effects.
  \item But in equilibrium: $P^d = P^s = P$, $Y^d = Y^s = Y$. So you do not specifically observe the two equaitons above!
  \item You are now observing an equilirbium point / a series of equilibriu points (say $(Y,P$)) that is JOINTLY determined by S and D,
  \item The relaiton between y adn p that you are observing is totally ininformative as you have 2 unobservables (D and S) determining jus tone observable: price may be determined by a symultaneous shift in D amd supply.
  \begin{figure}[H]
      \centering
      \includegraphics[width=0.55\linewidth]{Structural Model.png}
      \label{fig:enter-label}
  \end{figure}
  \item So causal interpretation of $\frac{\partial Y}{\partial P}$ is invalid due to endogeneity (simultaneity bias).
  \item You want to isolate shifts in D or S
\end{itemize}

\subsection{Using Exogenous Variation: Reduced Form}

\begin{itemize}
  \item If we can vary $Z^d$ and/or $Z^s$ exogenously, we can recover causal effects (IV style).
  \item Reduced form (endogenous as a function of exogenous variables):
  \[
  \begin{aligned}
  Y &= Y(Z^d, U^d, Z^s, U^s) \\
  P &= P(Z^d, U^d, Z^s, U^s)
  \end{aligned}
  \]
  \item If $Z^s$ affects $P$ and can be exogenously varied (no effect on $Z^d, U^d, U^s$), we can identify causal effect of $Z^s$ on $P$ and $Y$.
  \item idea is shiftng the suppyll curve while keeping the demand curve fixed
  \item Reduced Form Causal Effects of $Z^s$ on P and Y (ceteris paribus):
  \[
  \frac{\partial Y}{\partial Z^s}, \quad \frac{\partial P}{\partial Z^s}
  \]
\end{itemize}

\section{Structural Model: Concrete Example}

\subsection{Model Setup}

\begin{itemize}
  \item Structural equations:
  \[
  \begin{aligned}
  Y^d &= \beta_0 + \beta_1 P^d + U \\
  Y^s &= P^s - Z
  \end{aligned}
  \]
  \item In equilibrium: $P^d = P^s = P$, $Y^d = Y^s = Y$
  \item Structural form:
  \[
  \begin{aligned}
  Y &= \beta_0 + \beta_1 P + U \\
  P &= Y + Z
  \end{aligned}
  \]
  we cannot identify the structural coefficient $\beta_1$ representing the ceteris paribus marginal effect of $P^d$ on $Y^d$
\end{itemize}


\subsection{Solving the System}

\begin{itemize}
    \item Assumption: U and Z are independent!
  \item Solve the system, reduced form of the model:
  \begin{align*}
    Y &= \frac{\beta_0}{1 - \beta_1} + \frac{\beta_1}{1 - \beta_1} Z + \frac{U}{1 - \beta_1} \\
    P &= \frac{\beta_0}{1 - \beta_1} + \frac{1}{1 - \beta_1} Z + \frac{U}{1 - \beta_1}
    \end{align*}
  \item Let $\gamma_0 = \frac{\beta_0}{1 - \beta_1}, \gamma_1 = \frac{\beta_1}{1 - \beta_1}, \gamma_2 = \frac{1}{1 - \beta_1}$
  \item Reduced form causal effects:
  \[
  \frac{\partial Y}{\partial Z^s} = \gamma_1, \quad \frac{\partial P}{\partial Z^s} = \gamma_2
  \]
  \item Recover structural parameter of interest from the reduced parameters:
  \[
  \frac{\partial Y^d}{\partial P^d} = \beta_1 = \frac{\partial Y / \partial Z^s}{\partial P / \partial Z^sw}
  \]
  The Exogenous variation in Supply allowed you to find the slope of Demand
\end{itemize}

\subsection*{Interpretation}

\begin{itemize}
  \item $\gamma_1 = \frac{\beta_1}{1 - \beta_1}$ is causal effect of $Z^s$ on $Y$.
  \item Reduced form coefficient is the composition of two effects:
  \[
    \frac{\partial Y}{\partial Z} = \frac{\partial Y^s}{\partial P^s} \cdot \frac{\partial P}{\partial Z}
    = \beta_1 \cdot \frac{1}{1 - \beta_1}.
   \]
Z is the exogenous variable generating the change in Y, while P is the intermediate cause.
\end{itemize}

\section{Structural Models: Other Examples}

\subsection{Identifying the Supply Slope}

\begin{itemize}
  \item New model:
  \[
  \begin{aligned}
  Y^d &= \beta_0 + \beta_1 P^d + Z + U \\
  Y^s &= \beta_2 P^s
  \end{aligned}
  \]
  \item Goal: identify $\beta_2$
  \item You can identify the slope of supply BECAUSE you have exogenous shocks in Demand!
\end{itemize}

\section{Structural Approach and Potential Outcomes}

\begin{itemize}
\item Assumptions needed to identify $\beta$ in a structural equation are equivalent to potential outcome assumptions.
  \item Structural equation $Y_i = \beta D_i + \varepsilon_i$ 
  \begin{itemize}
  \item Assumes homogeneous $\beta$ (guarantees no treatemtnegfect heteroegenity)
  \item Selection bias arises if $D_i$ is correlated with $\varepsilon_i$  (some unobserved characteristics are correlated with the treatement assignment)
  \end{itemize}
\end{itemize}















\end{document}

\documentclass[9pt,a4paper,twoside]{rho-class/rho}
\usepackage[english]{babel}
\let\bibhang\relax


\setbool{rho-abstract}{true} % Set false to hide the abstract
\setbool{corres-info}{true} % Set false to hide the corresponding author 
\usepackage{xcolor}
\usepackage{soul}
\definecolor{lightblue}{RGB}{135, 206, 250}
\usepackage{graphicx}  % in preamble
\usepackage{fancyhdr}
\usepackage{changepage}  % or geometry
\usepackage{graphicx}    % for resizebox
\usepackage{xcolor}

\usepackage{soul,color}
\usepackage{tabularx}
\definecolor{issue}{RGB}{220,50,47}       % red
\definecolor{example}{RGB}{0,150,0}      % green
\definecolor{intuition}{RGB}{108,113,196} % purple
\definecolor{solution}{RGB}{38,139,210}    % blue
\usepackage{amsmath}  % in preamble

%----------------------------------------------------------
% TITLE
%----------------------------------------------------------

\title{AP: Duflo (2001)}

%----------------------------------------------------------
% AUTHORS AND AFFILIATIONS
%----------------------------------------------------------

\author{Alessandro Caggia}

%----------------------------------------------------------
% DATES
%----------------------------------------------------------

\dates{June 2025}

%----------------------------------------------------------
% FOOTER INFORMATION
%----------------------------------------------------------

\institution{Bocconi University}
\theday{} %\today

%----------------------------------------------------------
% ABSTRACT
%----------------------------------------------------------


\begin{abstract}
\end{abstract}



\newcommand{\citeyearcomma}[1]{\citeauthor{#1}, \citeyear{#1}}

\begin{document}
    \maketitle
    \thispagestyle{plain}
    \linenumbers


%----------------------------------------------------------

\section{Application: Duflo (2001)}

\subsection{Research Questions}
\begin{itemize}
  \item Effect of school infrastructure on future wages (through higher education)
\end{itemize}

\subsection{The Natural Experiment}
\begin{itemize}
  \item Massive primary school construction program, started in 1973 in Indonesia: between 1973--1974 and 1978--1979 more than 60,000 schools were built.
\end{itemize}

\subsection{Data}
\begin{itemize}
  \item Uses 1995 Census, more than 150,000 individuals. This is A SINGLE CROSS SECTION. 
  \item Only 60,000 work for a wage (others are self-employed).
\end{itemize}

\subsection{Treatment Assignment}
\begin{itemize}
  \item Number of schools to be built proportional to number of unenrolled primary school-age children in 1972.
\end{itemize}

\subsection*{Sources of Exogenous Variation}
Brilliant: COHORT PLAYS THE ROLE OF TIME!
Two key sources:
\begin{enumerate}
  \item \textbf{District variation:} “High intensity” districts = more schools given population (residuals from school allocation regression).
  \begin{itemize}
          \item Duflo estimates the following school allocation regression:
    \[
    \log(\text{Schools}_d) = \alpha + \beta_1 \ln(\text{Children}_d) + \beta_2 \ln(1 - \text{Enr.Rate}_d) + \varepsilon_d
    \]
    
    Then, for each district \( d \), the residual \( \varepsilon_d \) captures how many more or fewer schools the district received compared to what the regression predicts:
    
    \begin{itemize}
        \item If \( \varepsilon_d > 0 \): the district received \textbf{more schools than expected}.
        \item If \( \varepsilon_d < 0 \): the district received \textbf{fewer schools than expected}.
    \end{itemize}
    
  \end{itemize}
  \item \textbf{Cohort variation:} children aged \textgreater12 in 1974 unaffected (already out of primary school); starting from 1974 The younger a child was in 1974, the more it benefited from the program –because spent more time in the new schools.
\end{enumerate}

\subsection*{Simple Difference 1: High vs Low Districts}
\begin{itemize}
\item Why not just estimate a simple difference: comparing schooling in high vs in low intensity districts?
  \item Biased: high-intensity districts had lower educaiton achievement pre-program. So T and C groups are not comparable (selection bias). But they idea is: they are diffrent there is selection bias but if I control for such selection we are fine. in this case you would need to control for a lott of unobservables and enrollment achievemtn pre-program (which we don't have a good measuer of).
\end{itemize}

\subsection*{Simple Difference 2: Young vs Older Children}
\begin{itemize}
  \item Young = 1 if age 2--6 in 1974, 0 if age 12--17.
\item Why not just estimate simple difference: comparing schooling for young vs older children? (Even without the program, younger cohorts typically get more schooling than older ones (due to economic growth, reforms, etc)
  \item Biased due to secular upward trend in education over time. If the problem had been: t1 cohort are more males, you control and yo uare fine. the problem is: t1 cohort has more educ, which is also your outcome. 
  \item you could use time effects to control for this but you need a lot of time periods
  \item Unbiased only if educational attainment would have evolved similarly for both groups absent the program ( education young cohort would have had without the program similar to education old cohort would have had without the program).
\end{itemize}


\subsection{Final methodology}
\begin{align*}
\text{DID} &= 
\textcolor{blue}{
\left[ \mathbb{E}(S \mid \text{Young} = 1,\ \text{High}) 
- \mathbb{E}(S \mid \text{Young} = 0,\ \text{High}) \right]} \\
&\quad -\ 
\textcolor{red}{
\left[ \mathbb{E}(S \mid \text{Young} = 1,\ \text{Low}) 
- \mathbb{E}(S \mid \text{Young} = 0,\ \text{Low}) \right]}
\end{align*}





\begin{itemize}
\item the second term in the difference is time! this is a normal DiD
    \item with the first term you remove the bias coming from being in better district
    \item with the secon dyou remove the secular trend
\end{itemize}

\subsection{Results}
exactly showing what we have discussed above thoererically- 
\begin{figure}[H]
    \centering
    \includegraphics[width=0.9\linewidth]{MM_5_RDD/duflo_results.png}
    \label{fig:enter-label}
\end{figure}

\begin{itemize}
    \item \textbf{Col}: difference due to the fact that high program areas are worse. So they gain +0.47 and we partialled out the intrinsically lower level (8.02 vs 9.40) 
    \item \textbf{Row}: difference due to overall increases in education within regions over time.  Out of 0.47 we remove 0.36 of common trend! 
    \item \textbf{DID}: effect of the program, differences across districts between 2–6 year-old children and 12–17 year-old children.
    \item 0.12 additional years of education on average, 0.026 difference in log wages.
    \item \textbf{Placebo test}: DID for two groups whom we know did not benefit --- the estimated effect is indeed statistically 0 (aged 12–17 in 1974 vs aged 18-24).
\end{itemize}

\subsection{Wald-DID Estimator: now we need to use the IV}

\[
\frac{\text{DID of } Y}{\text{DID of } D} = \frac{0.026}{0.12} = 0.217
\]

\begin{itemize}
    \item School construction as IV to estimate the returns to education.
    \item Equivalent to a 2SLS of \( Y \) (wages) o\textbf{n district dummies, cohort dummies, education attainment}, using the interaction of time and group dummies as instrument.
    \item de Chaisemartin and D'Haultfœuille (2016) show that in a heterogeneous treatment effect setting the\textbf{ Wald-DID estimator this gives a weighted average of Wald-DIDs}. The weights \( w_{g,t} \) depend on the strength of the first stage (i.e., how much the instrument shifts treatment) and the size of each group-time cell. Therefore, the 2SLS estimate gives more weight to comparisons where the instrument is more powerful, and thus identifies a weighted average of local treatment effects.
 
\end{itemize}

\paragraph{Regression Equivalent for Each DID with More Than Two Groups}
Practically, Duflo estimated the did through the following regression
\begin{align*}
\text{DID:} \quad & S_{ijk} = c + T_i \gamma + G_j \delta + (G_j \times T_i) \rho + \varepsilon_{ijk} \\
\end{align*}

\begin{itemize}
    \item \( S_{ijk} \): schooling (first stage) or wages (reduced form) for individual \( i \) in region \( j \), born in year \( k \).
    \item \( G_j \): dummy for being in high intensity district.
    \item \( T_i \): dummy for being in young cohort.
    \item \( a_j \): district of birth fixed effect, captures differences across districts (instead of \( G_j \)).
    \item \( \beta_k \): cohort of birth fixed effect, captures differences across cohorts (instead of \( T_i \)).
    \item \( C_j \): vector of region-specific variables for district \( j \).
    \item \( P_j \): intensity of the program in district of birth \( j \).
\end{itemize}


note how the results are lower  in table 4
\[
\frac{0.0270}{0.188} \approx 0.144
\]
\begin{table}[H]
\centering
\renewcommand{\arraystretch}{1.2}
\scriptsize % or \scriptsize for further shrinkage
\begin{tabular}{p{2cm} | p{2cm} | p{3.5cm}} % slightly narrower columns
\toprule
\textbf{Source of Difference} & \textbf{Table 3 (Basic DID)} & \textbf{Table 4 (Regression/IV)} \\
\midrule
\textbf{0) Functional form} &
Non Parametric &
Linear and additive, but can incorporate richer controls and continuous treatment \\
\midrule
\textbf{1) Controls for cohort (birth year)} &
Cohorts grouped (Young = 2--6, Old = 12--17) — coarse, no specific birth-year FE &
Includes full year-of-birth fixed effects ($\beta_k$) — fine-grained control \\
\midrule
\textbf{2) Program intensity measure} &
Binary high/low intensity dummy (discrete treatment variable) &
Continuous: number of schools per 1,000 children — captures variation in exposure intensity \\
\midrule
\textbf{3) Treatment variation} &
Group-level variation (young/old × high/low region); no within-region variation &
Individual-level variation in treatment intensity (based on region $\times$ year of birth), leading to finer comparisons \\
\midrule
\textbf{4) Controls for region and cohort differences} &
Only 4 group means; controls are coarse: region = high/low, cohort = young/old &
Includes full region fixed effects ($a_j$) and cohort fixed effects ($\beta_k$), absorbing all systematic region and cohort variation. Allows also for time varying controls! \\
\bottomrule
\end{tabular}
\end{table}



\subsection{Identification threats}

\begin{itemize}
    \item \textbf{Mean reversion bias.}\\
    School construction was targeted to regions with low past education levels. \textbf{Risk:} These regions may improve over time even without the program, biasing the estimated effect upward.\\
    \textit{Mitigation:} Control for pre-treatment characteristics (e.g., 1971 enrollment, sanitation); EVEN BETTER (what we are doing): compare cohorts within region (young vs old) to remove time-invariant region-level bias (if there was a rebound this has characterized both young and old! ). This is ok if reversion is in anything but education. 

    \item \textbf{Confounding program shocks.}\\
    The school construction coincided with national shocks (e.g., oil windfalls, suppression of school fees). National shocks are absorbed by cohort/time fixed effects!

    \item \textbf{GE effects.}\\
    SUTVA? (more educated all around, bad outcome for control bc mroe competition in the amrket) scalability? \\
    \textit{Risk:} likely depends on localized inpmlemtnation and labor mobility


\end{itemize}



\section*{Extensions: Cohort-Specific Estimation (Event Study)}

\begin{itemize}
    \item \textbf{Estimate effects for each cohort separately.} \\
    No need to define arbitrary “young vs. old” groups. Instead, interact program intensity with cohort dummies.
\end{itemize}

\vspace{0.5em}
\begin{equation*}
S_{ijk} = c + a_j + \beta_k + \sum_{l=2}^{23} (P_j \cdot d_{il}) \gamma_l + \sum_{l=2}^{23} (C_j \cdot d_{il}) \delta_l + \varepsilon_{ijk}
\end{equation*}

\begin{figure}[H]
    \centering
    \includegraphics[width=0.7\linewidth]{MM_5_RDD/jcjcjcj.png}
    \caption{Coefficient on interactions}
    \label{fig:enter-label}
\end{figure}



\section*{Duflo (2001): Returns to Education --- Identification Threats}

\subsection*{Exclusion Restriction Assumption}

\begin{itemize}
    \item \textbf{Key requirement:} The instrument (school construction intensity $\times$ cohort exposure) must affect wages \textit{only through} education (enrollment: \hl{the goal is to estimate the causal effect of years of education on wages, using school construction as an instrument}). 
    \item \textbf{Violation concern:} If school construction affects wages through other channels (e.g., school quality, local economic development), the exclusion restriction fails. \hl{WHATEVER THAT IS NOT SCHOOLING YEARS}
\end{itemize}

\subsubsection*{1. School Construction May Affect School Quality}

\begin{itemize}
    \item \textbf{Mechanism:} The program may increase:
    \begin{itemize}
        \item Teacher hiring,
        \item Class size reduction,
        \item Classroom quality or resources,
        \item Peer quality.
    \end{itemize}
    \item \textbf{Risk:} These effects may increase wages \textit{independently} of education levels.
    \item \textbf{Implication:} Violates the exclusion restriction; bias in IV estimate.
\end{itemize}

\subsubsection*{Empirical Check: DID on School Quality Measures}

\begin{itemize}
    \item Duflo examines whether the program changed \textbf{teacher/pupil ratios} using a difference-in-differences (DID) design.
    \item \textbf{Result (Table 6):} No significant effect of the program on teacher/pupil ratios.
    \item \textbf{Interpretation:} Limited evidence that school quality improved alongside school quantity.
\end{itemize}

\subsubsection{Placebo Tests on Subpopulations Without Education Margin}

\begin{itemize}
    \item \textbf{Strategy:} Identify subgroups for whom school construction should \textit{not} affect education quantity (exckyde tgat there was a braod effect of ecoomic devekopmen etc):
    \begin{itemize}
        \item Individuals with more than 9 years of schooling (already beyond primary),
        \item Individuals in dense districts (already high school access).
    \end{itemize}
    \item \textbf{Finding:} No wage effects in these subsamples.
    \item \textbf{Implication:} Suggests the wage effect operates \textit{only} via increased education.
\end{itemize}

\subsection*{Selection Bias: Full vs Observed Job Market}
Heckman selection model: our sample is made up of self selected individuals. In our case: individuals on average 
If we observed wages for the full population (including self-employed and non-wage workers), the DID would correctly capture the average wage effect of education for all individuals. In our case, however, wages are observed only for wage earners. This marekt is made up by the most educated people. Before treatment it was made up by the most educated people in the T group and in the C group (more skilled? more motivated). Now, compliers among the treated gain extra educaiton and enter the market. This expands the observed sample to include marginal individuals in the treated group, while the control group remains positively selected. Some of the marginal individuals are entering bc ceteris paribus in skills they lacked educ, because now they compensate low skill with more educ! \textbf{THE CONTROL GROUP IS NOT TH EVALID CONTERFACTUAL OF THE TRETAMETN GROUP!!!!}

\begin{figure}[H]
    \centering
    \includegraphics[width=0.8\linewidth]{MM_5_RDD/SAMPLE_SELECTION.png}
    \label{fig:enter-label}
\end{figure}
Need (reasonable) assumptions on the distirbution of human capital. The rest is just pure intuition. \textbf{Expect downward Bias!!!!}

\end{document}
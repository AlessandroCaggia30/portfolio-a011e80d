\documentclass[9pt,a4paper,twoside]{rho-class/rho}
\usepackage[english]{babel}
\let\bibhang\relax


\setbool{rho-abstract}{true} % Set false to hide the abstract
\setbool{corres-info}{true} % Set false to hide the corresponding author 
\usepackage{xcolor}
\usepackage{soul}
\definecolor{lightblue}{RGB}{135, 206, 250}
\usepackage{graphicx}  % in preamble
\usepackage{fancyhdr}
\usepackage{changepage}  % or geometry
\usepackage{graphicx}    % for resizebox
\usepackage{xcolor}

\usepackage{soul,color}
\usepackage{tabularx}
\definecolor{issue}{RGB}{220,50,47}       % red
\definecolor{example}{RGB}{0,150,0}      % green
\definecolor{intuition}{RGB}{108,113,196} % purple
\definecolor{solution}{RGB}{38,139,210}    % blue
\usepackage{amsmath}  % in preamble

%----------------------------------------------------------
% TITLE
%----------------------------------------------------------

\title{More IVs}

%----------------------------------------------------------
% AUTHORS AND AFFILIATIONS
%----------------------------------------------------------

\author{Alessandro Caggia}

%----------------------------------------------------------
% DATES
%----------------------------------------------------------

\dates{June 2025}

%----------------------------------------------------------
% FOOTER INFORMATION
%----------------------------------------------------------

\institution{Bocconi University}
\theday{} %\today

%----------------------------------------------------------
% ABSTRACT
%----------------------------------------------------------


\begin{abstract}
\end{abstract}



\newcommand{\citeyearcomma}[1]{\citeauthor{#1}, \citeyear{#1}}

\begin{document}
    \maketitle
    \thispagestyle{plain}
    \linenumbers


\section{Shift-Share Instruments (Bartik IVs)}

\subsection*{Classic Endogeneity Problem}

\begin{itemize}
    \item Recall the standard endogeneity issue in market equilibrium models: prices and quantities (e.g., wages and employment) are determined simultaneously.
    \item To identify the slope of the supply curve, we need valid demand shifters (i.e., instruments that shift demand but are exogenous to supply).
\end{itemize}

\subsection*{Estimation Equation}

\begin{itemize}
    \item The equation of interest is:
    \[
        w_\ell = \beta N_\ell + \gamma X_\ell + \varepsilon_\ell \tag{1}
    \]
    \item Where:
    \begin{itemize}
        \item \( w_\ell \): wages in region \( \ell \)
        \item \( N_\ell \): employment in region \( \ell \)
        \item \( X_\ell \): controls (e.g., demographics, amenities)
        \item \( \varepsilon_\ell \): unobserved error term
    \end{itemize}
    \item Endogeneity arises because \( N_\ell \) is jointly determined with \( w_\ell \), so we instrument \( N_\ell \) with \( z_\ell \) $\rightarrow$ jointly deteried by demand and supply. We are creating now an exogenous shock in demand.
\end{itemize}



\subsection*{The Bartik Instrument}

\begin{itemize}
    \item Bartik (1991) constructs a shift-share IV by predicting local demand shocks using national industry growth, weighted by local industrial composition.
    \item The instrument is:
    \[
        z_\ell = \sum_k s_{\ell k} g_k \tag{2}
    \]
    \item Where:
    \begin{itemize}
        \item \( \ell \): index for local or regional labor markets
        \item \( k \): index for industries/sectors
        \item \( s_{\ell k} \): the share of industry \( k \) in local market \( \ell \), typically measured at a baseline year (e.g., initial period)
        \item \( g_k \): the national growth rate of industry \( k \), optionally leave-one-out to avoid local endogeneity: \( g_{k, -\ell} \)
    \end{itemize}
    \item \textbf{Key interpretation:} The Bartik IV is a weighted sum of national industry shocks (\( g_k \)), where the weights are the initial shares of each sector in the local economy (\( s_{\ell k} \)). This generates region-specific shocks even though the national shocks are common across regions.

\end{itemize}
\subsection*{Validity Conditions}

To be a valid IV, \( z_\ell \) must satisfy two conditions:

\begin{enumerate}
    \item \textbf{Relevance:} \( z_\ell \) must be correlated with \( N_\ell \)
    \item \textbf{Exogeneity:} \( z_\ell \) must be uncorrelated with \( \varepsilon_\ell \)
\end{enumerate}

\textbf{For Bartik IV, this means: IT HAS TO BE AN EXOGENOUS DEMAND SHOCK}
\begin{itemize}
    \item \textbf{1. National shocks \( g_k \)} must be exogenous to local shocks:
    \[
        g_k \perp \varepsilon_\ell \quad \text{for all } \ell
    \]
    \item \textbf{2. Local past distribution of employment across sector  \( s_{\ell k} \)} exogenous to local contemporaneous shocks:
    \[
        s_{\ell k} \perp \varepsilon_\ell
    \]
\end{itemize}

\subsection*{Exclusion Restriction?}
\begin{itemize}
    \item Key question: Do either the shares \( s_{\ell k} \) or the shocks \( g_k \) affect outcomes directly, outside of their effect on \( N_\ell \)?
    \item If, for instance, local shares \( s_{\ell k} \) were chosen in anticipation of future shocks, or \( g_k \) reflects national trends influenced by local economic conditions, the exclusion restriction would be violated.
\end{itemize}


\subsection{Applications of Shift-Share IVs}

\begin{itemize}
    \item Bartik IVs became more popular after:
    \begin{itemize}
        \item \textbf{Autor et al. (2013):} Study of the \textit{China shock}: they exploit heterogeneity in industrial composition across US regions and interact it with national growth in Chinese imports.
        \item \textbf{Card} on immigrants
    \end{itemize}
\end{itemize}


\subsection{Identification: Goldsmith-Pinkham et al. (2020)}

\begin{itemize}
    \item Their key insight: identification depends not just on exogeneity of \( g_k \), but also on exogeneity of \( s_{\ell k} \)
    \item Each \( s_{\ell k} \) can be seen as generating a separate instrument for \( \ell \). Each sector (defined by its share) experiences a different change in global demand (global migration) = different local evolution in demand (first stage: predicted change in migration, first stage setting wrt actual change in migration). The total Bartik IV estimate is a weighted average of these sector-level IV estimates (Roemberg weight).
    \[
\hat{\beta}^{\text{Bartik}} = \sum_k \hat{\alpha}_k \hat{\beta}_k, \quad \text{where} \quad \sum_k \hat{\alpha}_k = 1
\]

    \item Robustness: estimate separate IV regressions using each \( k \)-level component.
    \item Each \( k \)-level IV contributes to the total estimate with a **Rotemberg weight** \( \alpha_k \), defined as:
    \[
        \alpha_k = \frac{g_k Z_k' N}{\sum_k g_k Z_k' N} \tag{3}
    \]
    where \( Z_k \) is the vector of shares of sector \( k \) across all regions. The larger the global shock, the more the industry is \textbf{heterogeneously} present across the regions (Zk varies a lot), and the more variation in Zk is correlated with variation in employment N (component of the first stage) the greater the weight
   

    \item \textbf{Diagnostics:}
    \begin{itemize}
        \item Negative Rotemberg weights are problematic. Negative weigth sdpend on the sign of the product $g_k Z_k' N$. You have a negative weight if $g_k$ (global expansion in export for sector k) and $Z_k' N$ move in opposite directions: $g_k$ is saying: there was a positive shock for sector k (manufacturing), g > 0. \\ $Z_k' N$ is saying (if < 0): there is a negative relationship between the local PREDICTED exposure to industry k (manufacturing) and employment in the area (the endogenous areas). The more the exposure to industry k the lower employment \\
        The sector’s shock says: “employment should go up”. But: regions more exposed to that sector actually have lower employment.\\
        \textcolor{violet
        }{We have an underlying assumption that global g_k is positively or negatively (but one of the two) correlated with global N}
        \item check fraction of negative weights is not too high
        \item Identification narrative and F-statistics (weak instruments?) for the k dimensions with largest Rotemberg weights
        \item \textbf{Sectors with large weights} should have \( \hat{\beta}_k \) close to \( \hat{\beta}^{bartik} \); otherwise, result is sensitive to few sectors.
        
    \end{itemize}
    \item Borusyak et al. (2021) offer an alternative: focus on exogeneity of the shocks \( g_k \) instead of shares \( s_{\ell k} \).
\end{itemize}




\section{IV in Randomized Trials}

\begin{itemize}
    \item The instrument is a dummy for being assigned to treatment.
    
    \item The effect of assignment (the instrument) on outcomes is the \textbf{ITT}, an effect of interest in many cases (e.g., effect of being offered a free bank account).
    
    \item That is all we can get if the exclusion restriction does not hold. Note, A comparison between treated and control is not valid if the sample opening the bank account is not random.
    
    \item The IV estimate, \texttt{ivregress 2sls y (D = Z)}, is equal to the ratio between the ITT and the difference in compliance rates between treatment and control 
    (recall previous lecture slides on IV and LATE):
    \[
    \text{LATE} = \frac{\text{ITT}}{\Pr(D = 1 \mid Z = 1) - \Pr(D = 1 \mid Z = 0)}
    \]
    
    \item If compliance among the controls is perfect (no single control is treated), then the IV is the ATT. Rceall ex. 3 of interpretation of LATE! LATE  is ATT if all people being offered the treatemtn take it up (I mean \textbf{recall the first set of notes on rubin modle, compliers are thsoe that are treated! (tthose that when treated react by taking up the treatmetn!} here the two correspond!)
\end{itemize}

\vspace{1em}
\subsection*{Example: IV in Randomized Trials}

\begin{itemize}
    \item Impact of having a bank account, using as instrument the assignment to being offered a free savings account.
    
    \item \textbf{Data:}
    \begin{itemize}
        \item Average savings for those offered a free savings account: \$2,500
        \item Average savings for those not offered a free savings account: \$1,500
        \item Proportion of those offered who opened the account: 50\%
        \item Proportion of those not offered who opened the account: 20\%
    \end{itemize}
    
    \item \textbf{Calculations:}
    \begin{itemize}
        \item ITT = \$2,500 - \$1,500 = \$1,000
        \item LATE = 
        \[
        \frac{2,500 - 1,500}{0.5 - 0.2} = \frac{1,000}{0.3} = 3,333.33
        \]
    \end{itemize}
    
    \item \textbf{Special case (perfect compliance in control group):}
    \begin{itemize}
        \item If proportion of those \textbf{not offered} who opened an account is 0\%, then:
        \[
        \text{LATE} = \frac{2,500 - 1,500}{0.5} = \frac{1,000}{0.5} = 2,000
        \]
        \item In this case: LATE = ATT
    \end{itemize}
\end{itemize}















\end{document}
